\documentclass[12pt]{scrartcl}

\usepackage{listings}
\usepackage{color}

\lstdefinestyle{input}{
  frame=lines,
  aboveskip=3mm,
  belowskip=3mm,
  basicstyle={\small\ttfamily},
}

\lstdefinestyle{c++}{
  language=C++,
  frame=lines,
  aboveskip=3mm,
  belowskip=3mm,
  basicstyle={\small\ttfamily},
  keywordstyle=\color{blue}\ttfamily,
  stringstyle=\color{green}\ttfamily,
}

\title{Aufgabe 2 --- Oma's Weihnachtskarten}
\subtitle{Dokumentation}
\author{Julius Makowski}
\date{}

\begin{document}
  \maketitle

  \section{Einleitung}
    Die programmatische L{\"o}sung f{\"u}r diese Aufgabe ist in C++ geschrieben und basiert auf dem in der Theorie beschriebenen Ansatz.

  \section{L{\"o}sung}
    \subsection{Eingabe / Ausgabe}
      Das Format der Eingabe sowie der Ausgabe entsprechen dem in der Aufgabe vorgegebenen Format.

      \vspace{12pt}\noindent
      Die Eingabe erfolgt dabei durch den stdin der Konsole.

      \vspace{12pt}\noindent
      Beispiele k{\"o}nnen im Programmordner gefunden werden.

    \pagebreak

    \subsection{Funktionsweise}
      Das Programm lie{\ss}t zuerst alle \textit{n} Senioren ein. Dabei nutzt es eine Zuornungstabelle, um dem jeweiligen Namen eine ID von einem leichter nutzbaren Datentyp (\textit{Integer}) zuzuordnen.
      
      \vspace{12pt}
      \begin{lstlisting}[style=c++]
      int n, m;
      cin >> n >> m;
      
      unordered_map<string, int> ids;
      for (int i = 0; i < n; i++)
      {
        string name;
        cin >> name;
        ids[name] = i;
      }
      \end{lstlisting}

      \vspace{12pt}\noindent
      Anschlie{\ss}end lie{\ss}t es alle \textit{m} Kanten ein und speichert diese in einem \textit{Array}. Daf{\"u}r nutzt es eine simple Datenstruktur \textit{Kante}, in der \textit{Quelle}, \textit{Ziel} sowie \textit{Gewicht} der jeweiligen Kante gespeichert werden.

      \vspace{12pt}
      \begin{lstlisting}[style=c++]
      struct Kante
      {
        int quelle, ziel, gewicht;
      };
      \end{lstlisting}

      \vspace{12pt}
      \begin{lstlisting}[style=c++]
      vector<Kante> graph;
      for (int i = 0; i < m; i++)
      {
        string quelle, ziel;
        int gewicht;
        cin >> quelle >> ziel >> gewicht;

        graph.push_back({ids[quelle], ids[ziel], gewicht});
      }
      \end{lstlisting}

      \pagebreak

      \vspace{12pt}\noindent
      Nun kann der tats{\"a}chliche Algorithmus gestartet werden. Daf{\"u}r werden zuerst Variablen zum Speichern der L{\"o}sung, des Grades der aktuell bearbeiteten Kante sowie eine Union-Find-Struktur f{\"u}r die \textit{n} Knoten.

      \vspace{12pt}
      \begin{lstlisting}[style=c++]
      int besteSumme = numeric_limits<int>::max();
      vector<Kante> loesung;
      
      vector<int> grad(n);
      grad[0] = 1;
      
      UnionFind uf(n);
      
      solve(graph, 0, 0, loesung, grad, uf, besteSumme);
      \end{lstlisting}

      \vspace{12pt}\noindent
      In der \textit{solve} Methode wird zuerst der R{\"u}ckkehrpunkt f{\"u}r die Funktion definiert. Dies tritt ein, wenn der letzte Knoten bearbeitet wurde, also \textit{index} gleich \textit{n}. Wenn die Anzahl der Kanten in der L{\"o}sung der Gesamtanzahl an Knoten entspricht, ist auf dem Weg keine L{\"o}sung m{\"o}glich, das Gewicht der L{\"o}sung wird auf $\infty$ gesetzt. Wenn die L{\"o}sung ein geringeres Gewicht als die bisher beste L{\"o}sung hat, wird sie als neue beste L{\"o}sung deklariert.

      \vspace{12pt}
      \begin{lstlisting}[style=c++]
      if (index == kanten.size())
      {
        if (loesung.size() + 1 < uf.kanten.size())
          summe = numeric_limits<int>::max();
        if (besteSumme > summe)
          besteSumme = summe;
        return;
      }
      \end{lstlisting}

      \vspace{12pt}\noindent
      Anschlie{\ss}end wird {\"u}berpr{\"u}ft, wie fortgeschritten werden muss. Dazu wird sichergegangen, dass in der dezeitigen L{\"o}sung noch keine Kante zum Ziel der aktuellen Kante f{\"u}hrt (\textit{grad = 0}). Weiterhin d{\"u}rfen sich Quelle und Ziel nicht in der gleichen Gruppe der Union-Find-Struktur befinden - in diesen F{\"a}llen wird einfach mit der n{\"a}chsten Kante fortgefahren. Andernfalls werden Quelle und Ziel der gleichen Gruppe hinzugef{\"u}gt, der Grad des aktuellen Ziels erh{\"o}ht, das Gewicht der aktuellen Kante zur Gesamtsumme hinzugef{\"u}gt und anschlie{\ss}end probiert, mit diesen Parametern eine neue, bessere L{\"o}sung zu finden. 

      \vspace{12pt}\noindent
      Die letzendlich beste gefundene L{\"o}sung wird ausgegeben.

\end{document}
