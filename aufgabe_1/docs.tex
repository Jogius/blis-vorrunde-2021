\documentclass[12pt]{scrartcl}

\usepackage{listings}
\usepackage{color}

\lstdefinestyle{input}{
  frame=lines,
  aboveskip=3mm,
  belowskip=3mm,
  basicstyle={\small\ttfamily},
}

\lstdefinestyle{c++}{
  language=C++,
  frame=lines,
  aboveskip=3mm,
  belowskip=3mm,
  basicstyle={\small\ttfamily},
  keywordstyle=\color{blue}\ttfamily,
  stringstyle=\color{green}\ttfamily,
}

\title{Aufgabe 1 --- Ameisen}
\subtitle{Dokumentation}
\author{Julius Makowski}
\date{}

\begin{document}
  \maketitle

  \section{Einleitung}
    Die programmatische L{\"o}sung f{\"u}r diese Aufgabe ist in C++ geschrieben und basiert auf dem in der Theorie beschriebenen Ansatz.

  \section{L{\"o}sung}
    \subsection{Eingabe / Ausgabe}
      Da zu der Aufgabe keine genaueren Vorgaben gegeben wurden, habe ich ein eigenes Konzept zur Eingabe sowie Ausgabe der L{\"o}sung entworfen. Dieses wird im Folgenden beschrieben.
      
      \vspace{12pt}\noindent
      Die Eingabe erfolgt durch den stdin der Konsole. Allgemein ist der Input so aufgebaut:

      \vspace{12pt}
      \begin{lstlisting}[style=input]
      <Breite in cm> <Anzahl Ameisen>
      <Standort in cm> <Blickrichtung (0 = Rechts, 1 = Links)>
      ...
      ... Anzahl Ameisen
      ...
      <Standort in cm> <Blickrichtung (0 = Rechts, 1 = Links)>
      \end{lstlisting}

      \vspace{12pt}\noindent
      Beispiele k{\"o}nnen im Programmordner gefunden werden.

      \vspace{12pt}\noindent
      Die Ausgabe erfolgt als eine einzelne Zeile mit der Testdauer in Sekunden.

    \pagebreak

    \subsection{Funktionsweise}
      Das Programm errechnet beim Einlesen des Inputs den Abstand jeder Ameise zu der ihr gegen{\"u}berliegenden Kante
      
      \vspace{12pt}
      \begin{lstlisting}[style=c++]
      int currentDistance = direction ? position
                                      : width - position;
      \end{lstlisting}

      \vspace{12pt}\noindent
      und vergleicht diesen mit dem aktuell h{\"o}chsten ermittelten Abstand

      \vspace{12pt}
      \begin{lstlisting}[style=c++]
      if (aktuellerAbstand > maximaleDistanz)
        maximaleDistanz = aktuellerAbstand;
      \end{lstlisting}

      \vspace{12pt}\noindent
      Dadurch wird letzendlich die h{\"o}chste Distanz ermittelt und die Dauer des Tests berechnet und ausgegeben.

      \vspace{12pt}
      \begin{lstlisting}[style=c++]
      int time = maxDistance * 0.6;
      cout << time << endl;
      \end{lstlisting}

\end{document}
