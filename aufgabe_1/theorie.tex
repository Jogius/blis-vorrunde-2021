\documentclass[12pt]{scrartcl}

\title{Aufgabe 1 --- Ameisen}
\subtitle{Theorethische L\"osung}
\author{Julius Makowski}
\date{}

\begin{document}
  \maketitle

  \section{Ansatz}
    Zum besseren Verst{\"a}ndnis kann man bei dieser Aufgabe davon ausgehen, dass die Ameisen nicht aufeinandertreffen sondern sich durchqueren k{\"o}nnen.

  \section{L{\"o}sung}
    Mit diesem einfachen Ansatz ist sofort erkenntlich, dass die Zeit, bis alle Ameisen vom Draht gefallen sind, von der Anzahl \(n\) der Ameisen unabh{\"a}ngig ist.

    \vspace{12pt}\noindent
    Der einzige Faktor der in diese Zeit spielt ist somit die Anfangsposition der Ameisen. Bei einer Geschwindigkeit von \(1\frac{m}{min}\) kann die Zeit f{\"u}r eine Ameise mit

    \vspace{12pt}
    \(t = d * \frac{6}{10}\)

    \vspace{12pt}\noindent
    berechnet werden, wobei d der Abstand in Zentimeter zur Kante (Blickrichtung zu Beginn) und t die Zeit in Sekunden ist.

    \vspace{12pt}\noindent
    Die maximale Dauer eines Tests sind somit 60s.

\end{document}
